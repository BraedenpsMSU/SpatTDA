%TODO write make file
%TODO maybe not focused enough
    Prostate Cancer (PCa) is the second leading cause of cancer-deaths for men in the United States and is expected to make
    up 20\% of new cancer diagnosis in men in 2019\cite{siegel2019cancer}.
    The Gleason grading system is currently the primary system for grading PCa's architecture and stands as one of the
        most important predictors of prognosis.
    The Gleason system divides the multiple architectural patterns of Prostate cancer into 5 categories.
    The Gleason score is assigned to a whole slide image of a prostate biopsy notating which of these two of these two
        architectural patterns groups are most prevalent in the sample.
    However, the gleason score also suffer from high intraobserver and interobserver variability when being assigned by
    a Pathologist who must visually inspect the cancer slides.\cite{walsh2013gleason}
    This produces as need for new effective quantitative measures on prostate cancer.

    The emergence of digital pathology has enabled researchers to examine digital histopathological slides with greater
        detail.
    Two areas that have been applied to Histopathological data are topological data analysis (TDA) and Spatial analysis.
    % sources here
    Both of these methods aim to examine the spatial properties of data but differ in their focus.
    TDA is uniquely suited to examining the shape of data while spatial analysis focuses mainly\com{a bit of strong word} on questions of
    spatial dependence, spatial association, and spatial heterogeneity.\cite{de_smith_geospatial_2018}


    % TODO Fix this section, If this todo is not enough then hopefully the orange text in the output is.
    \com{WIP: mainly a list right now - adapted from notes. need to clean up and add citations}
    Spatial Analysis has been used to examine the tumor mircoenvironment from an ecological perspective, examining cancer
        and immune cell co-location and relating it to patient survival in estrogen receptor negative-breast cancer.
    Similar work has shown that statistical hotspot analysis correlates with better prognosis as well.
    Other studies demonstrated measures of spatial homogeneity in stromal cells was associated with good outcome for
        patients with estrogen negative breast cancer patients.
    Combinations of spatial statistics on CD68+ macrophages in human head and neck tumors have also been shown
        accurately match human predictions. % This not the TDA paper.

    TDA was successfully applied to breast cancer to identify subgroup of estrogen receptor positive breast cancer which
        had an 100\% survival rate. %TODO site - this should the gunnarson paper.
    %TODO you really want to abbrev. Persitent Homology?
    Persistent Homology (PH) is a specific method from TDA which has found great success when applied to cancer data.
    A more detailed description of PH will be put off for now, and we will start with just describing some of its notable
        applications when examining Histopathology.

    %TODO cut this down to what is most relavent.
    PH has also been applied to subtypes of breast cancer data where it was shown to be able to identity breast cancer
        subtypes using the information provided by the distance between the nuclei in the sample cell.
    On other breast cancer data, PH was able to use the tumor mirco-environmental to identify breast cancer subtypes as
        well as predict patient survival.
    PH is uniquely well suit to be applied to PCa as the gleason scale is based entirely on the cancer's architectural
        pattern.
    Here is has been show that PH can cluster a fine collection of prostate architectural pattern then used in producing
        the gleason score.
    PH has also been demonstrated to predict prostate cancer aggressiveness using machine learning models.



