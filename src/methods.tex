\subsection*{Mantel Analysis based Methods}
    Mantel Analysis \com{check against langauge in \cite{dutilleul2011spatio}} are a set of methods base around the
    Mantel statistics and are focused on assessing the linear association between two square distance matrices
        of same dimension.

    Given two distance matrices $X$ and $Y$ the Mantel statistic is defined to be dot product the upper triangular
    portion of the distance matrices as vectors:
    \[z_{m} =  \sum_{i=0}^{n} \sum_{j=i+1}^{n} X_{i,j}Y_{i,j}\]
    Likewise, there is also the standardized Mantel statistics which uses the Pearson's $r$ statistic instead:
    \[r_{m} =   \frac{1}{n(n-1)/2-1} \sum_{i=0}^{n} \sum_{j=i+1}^{n}
                (X_{i,j}-\hat{\mu}_X)(Y_{i,j}-\hat{\mu}_Y)/\hat{\sigma}_X\hat{\sigma}_Y.
    \]

    Mantel statistics are primarily used in the context of Hypothesis testing where it is applied in two
        related methods called the Mantel and partial Mantel test.
    Here we will only be focusing only on the Mantel test.
    The Mantel test is a permutation test that typically uses either Mantel statistic or the standardized Mantel statistic as the
    test statistic and works under the assumptions that:
    \begin{enumerate}
        \item   The $i$th row and column both correspond to the same sample/location.
        \item   The samples are independent of one another.
    \end{enumerate}
    The null hypothesis for the Mantel test is that there is no linear association between the two matrices.
    The test itself is preformed by permuting the rows and columns of one matrix while keeping the other fixed and recomputing
        the test statistic for each permutation to form a reference distribution.
    The value test value is then compared to the reference distribution to obtain a p-value.

A more interesting use of the Mantel methodology and test statistic is the Mantel correlogram.
The mantle correlogram is an innovative method to examine spatial correlation use in only distance matrices.
The base procedure is performed by first dividing the range of the values of the geographic distance matrices into $N$ distance classes of equal size representing an interval of values in the geographic distance matrix.
Then the $K$th distance class, we produce a model matrix $M(k)$ where an entry has the value $K$ if it falls into the $K$th distance class, otherwise it is set to 0 in $M(k)$.
Then the standardized Mantel statistic between the model matrix and the response matrix is computed and the Mantel test procedure is performed.
This forms a graph of Mantel statistics with respect to the distance classes which is called the Mantel correlogram.
The interpretation of the resulting correlogram is then based on the shape of the significant values of the test.

%TODO talk about the simulation studies examining the Mantel correlogram on conditional gaussian simulation data.

%TODO talk about cancer grade/morphological Heterogenity and connect it back to Mantel correlogram.

%TODO talk about need to verify results to non-euclidean distances and any possible simulation studies suitable
% specific distance metric on PD or there summaries.

%TODO Modification to Mantel correlogram procedure to allow examination of multiple regions which have the same property.
%TODO decide on most meanful metric for comparisons.

%TODO move on to persistence summaries after and possibly autocorrelation functions (or any method that is not bound by hypothesis testing).